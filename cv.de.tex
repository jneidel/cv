\documentclass{cv}

\name{Jonathan Neidel}
\address{Geboren: {\em7. Aug 1997} in Berlin}
\address{Aktueller Stand: \em{Dez 2025}}
\address{\email{cv@jneidel.de} \\ \github[jneidel] \\ \homepage[jneidel.de]{https://jneidel.de} \\ \linkedin[jneidel]}

\begin{document}

\begin{rSection}{Beschreibung}

  \newcommand\skillWidth{0.48}
  \begin{minipage}[t]{\skillWidth\linewidth}

    Ich bin ein Senior Backend Software Engineer der strukturiert denkt, ehrlich kommuniziert, Probleme löst und seine eigenen Ideen einbringen will.

    \medskip

    Ich habe aus Neugier mit dem Programmieren begonnen.
    Was mich wirklich überzeugt hat, mich diesem Feld zu widmen, war das
    Potenzial eine Vielzahl von Problemen für mich und andere zu lösen.
    Für mich geht es beim Programmieren vor allem um das Lösen von technischen Problemen.

    \medskip

    Meine Programmier-Expertise kann in zwei sich ergänzende Teile aufgespalten werden.
    Da ist das autodidaktische experimentierende Hacking, angetrieben von dem
    Drang kreative Lösungen zu finden.
    Bestärkt wird dies durch die ausgebildete professionelle Handwerkskunst,
    welche von dem Wunsch getrieben wird sauberen Code in einer strukturierten
    kooperativen Weise zu schaffen.

    \medskip

    Vom Anfang an habe ich immer an meinen eigenen Projekten gearbeitet.
    Ich habe alles mögliche gebaut: APIs, ganze Webapps, kleine Scripte und vollumfängliche Kommandozeilen Apps.
    Die Kommandozeilen Umgebung liegt mir besonders am Herzen,
    denn die meiste meiner Arbeit findet im Terminal und Text Editor statt.

  \end{minipage}
  \hfill
  \begin{minipage}[t]{\skillWidth\linewidth}

    In der professionellen Software Entwicklung habe ich mehrjährige Erfahrung mit dem im integrierten
    Software Engineering Workflow (Die Arbeit in einer Ticket-basierten
    Umgebung, das Koordinieren mit und steuern von einem Team aus Entwicklern, PMs, QA und anderen
    internen und externen Stakeholdern.)
    Ich erstellte qualitative hochwertige Implementationen und kritische Bug-fixes in fristgerechter Weise.
    Dabei bringe ich auch immer selbstbestimmt Initiativen mit ein zum Verbessern oder Automatisieren der Prozesse und Umgebung, der technischen Dokumentation und der Entwickler-Experience.

    \medskip

    In diesen Tools liegen meine Stärken:
    
    \textbf{Programmiersprachen}
    \begin{itemize}
      \item Web: JavaScript, TypeScript, SASS, React
      \item Backend/General purpose: Node.js (Express), PHP (Symfony), Python
    \end{itemize}

    \textbf{Tools}
    \begin{itemize}
      \item Shell scripting (Bash)
      \item Git
      \item SQL und Nicht-SQL Datenbanken
      \item Jest, Webpack, Eslint
      \item GitLab CI, GitHub CI
      \item Linux, System Administrierung
      \item Docker, docker compose
      \item RabbitMQ
    \end{itemize}

  \end{minipage}
\end{rSection}

\begin{rSection}{Lebenslauf}
  \textbf{\href{https://www.endava.de/}{Endava} - Senior Backend Developer}
  \hfill {\em Feb 2022 - heute}

  Nach der Bachelorarbeit im Betrieb wurde ich als Entwickler angestellt und dann zum Senior Entwickler befördert.

  \textbf{\href{https://www.jobdirecto.com/}{Jobdirecto} - Lead Fullstack Developer}
  \hfill {\em Feb 2023 - Feb 2025}

  Einführung von Best Practices, Maintenance und Produktentwicklung.

  \textbf{HTW Berlin - B.Sc in Angewandter Informatik}, Note: 2,0 \hfill {\em 2019 - 2023}
  
  \href{https://github.com/jneidel/ba}{Bachelorarbeit: Entwicklung und Evaluation von Methoden zum Verbessern der Nutzbarkeit von CLI Apps}
  
  \textbf{\href{https://visit4.me}{Visit4me} - Junior Backend Developer}
  \hfill {\em 2019}

  Mitarbeit an der REST API und der Integration externer APIs.

  \textbf{Emil Molt Akademie - Fachabitur mit Fokus auf Wirtschaft}, NC: 1,3 \hfill {\em 2016 - 2018}

  \smallskip
    \textbf{\href{https://buchhaltungsbutler.de/}{Buchhaltungsbutler} - Praktikant}
  \hfill {\em Feb - Aug 2017}

\end{rSection}
\pagebreak

\begin{rSection}{IT Projekte}

  % \textbf{\href{}{}}
  % \hfill

  % \begin{list}{$\cdot$}{}
  %   \itemsep -0.5em \vspace{-0.5em}
  %   \smallskip
  %   \item
  % \end{list}

  \textbf{Endava: \href{https://tvthek.orf.at/}{ORF TVThek API}}
  \hfill
  {\em Feb 2023 - Sep 2025}

  PHP (Symfony, Sonata), Docker, GitLab CI, Technisches Schreiben, OpenAPI, RabbitMQ, PHPUnit, PHPStan, Webhooks, Nelmio, SQL (MariaDB, Doctrine), Jira
  \begin{list}{$\cdot$}{}
  \itemsep -0.5em \vspace{-0.5em}
    \smallskip
  \item \href{https://jneidel.com/dev/technical-documentation-with-arc42/}{Neuaufsetzen der technischen Dokumentation}.
  \item Modernisierung und Vervollständigung der API Dokumentation.
  \item Spezifierung und Leitung mehrerer Initiativen.
  \item Automatisierung des Deployment flows und der Changelog Generierung.
  \item Neuentwicklungen des lokalen Entwicklersetups.
  \end{list}

  \textbf{\href{https://www.jobdirecto.com/}{Jobdirecto}}
  \hfill
  {\em Feb 2023 - Apr 2025}

  Node.js (Express), React, MongoDB, Webpack, Stripe, Heroku, Jest
  \begin{list}{$\cdot$}{}
  \itemsep -0.5em \vspace{-0.5em}
    \smallskip
    \item Einführung von professionellen Software practices: Testing, technischer Dokumentation, Prozessstandardisierung, Backups und einer Staging Umgebung.
    \item Technische Leitung in der Strukturierung des Backlogs und Entwicklung neuer Features.
  \end{list}

  \textbf{Endava: \href{https://github.com/jneidel/oraclett}{oraclett}}
  \hfill
  {\em Dec 2022 - Aug 2023}

  Node.js (oclif), GitHub CI
  \begin{list}{$\cdot$}{}
  \itemsep -0.5em \vspace{-0.5em}
    \smallskip
  \item Programmierung einer vollumfänglichen und nutzerfreundlichen CLI App zur Automatisierung von Businessprozessen.
  \end{list}

  \textbf{Endava: Bewerber Management Tool}
  \hfill
  {\em Mar - Nov 2022}

  PHP (Symfony, Sonata), RabbitMQ, Docker
  \begin{list}{$\cdot$}{}
  \itemsep -0.5em \vspace{-0.5em}
    \smallskip
  \item Technische Leitung in der Implementierung einer Symfony Webapp.
  \item Entwicklung eines Microservices zum Versand von Emails basierend auf Message Queue Inputs.
  \item Projektmanagement und Abnahme von Pull Requests.
  \end{list}

  \textbf{Endava: Changelog Viewer}
  \hfill
  {\em Feb - Mar 2022}

  React, Docker, GitLab CI
  \begin{list}{$\cdot$}{}
  \itemsep -0.5em \vspace{-0.5em}
    \smallskip
  \item Schreiben von Parser und Generator zum Erzeugen einer interaktiven Variante des CHANGELOG.md.
  \item Aufsetzen von automatisierten Docker Deployments via Gitlab CI.
  \end{list}

\textbf{\href{https://github.com/htw-kbe-jneidel/main}{Komponenten-basierte Entwicklung - Abschlussprojekt}}
  \hfill
  {\em Dec 2021 - Feb 2022}

  TypeScript, Node.js (Express), RabbitMQ, MongoDB (Mongoose), GitHub
  \begin{list}{$\cdot$}{}
  \itemsep -0.5em \vspace{-0.5em}
    \smallskip
  \item Design und Implementierung einer Microservice Architektur, inkl. RPC Kommunikation via Message Queue.
  \end{list}
  \textbf{\href{https://github.com/vyvytn/avt}{DJ Tool}}: JavaScript, Web Audio API, Node.js (Express), Vue.js
  \hfill
  {\em 2021}
  

\textbf{\href{https://github.com/jneidel/what-to-do}{What to do}}: Node.js (Express), Vue.js, SQL (MariaDB), Jest, supertest
  \hfill
  {\em 2021}

  \textbf{\href{https://github.com/jneidel/yoga-vidya-thunderbird2csv-add-on}{Thunderbird2csv}}: Node.js (Express), Browser Extension
  \hfill
  {\em 2020}

  \textbf{\href{https://github.com/jneidel/lock-me-out}{Lock Me Out}}: TypeScript, Node.js (Express), GPG, Webpack, SASS, MongoDB, Eslint
  \hfill
  {\em 2019}

  \textbf{\href{https://github.com/jneidel/dotfiles}{dotfiles}}: Shell scripting, Linux, Shell, Vim, Emacs
  \hfill
  {\em seit 2018}

  \textbf{\href{https://github.com/jneidel/mangareader-dl}{mangareader-dl}}: Node.js, CLI, Web Scraping, TypeScript, TDD, Jest
  \hfill
  {\em seit 2018}

  \textbf{\href{https://github.com/jneidel/setup-webpack}{Setup Webpack}}: Webpack, JavaScript, SASS, npm package
  \hfill
  {\em 2018 - 2020}

 \textbf{\href{https://github.com/jneidel/jneidel.com}{jneidel.com Website}}: Hugo, JavaScript, Tailwind CSS, HTML, Netlify
  \hfill
  {\em seit 2017}

\end{rSection}
\pagebreak

\begin{rSection}{Sprachen}
  \begin{minipage}[l]{0.48\linewidth}
    {\bf Deutsch:} Muttersprachler
  \end{minipage}
  \begin{minipage}[l]{0.48\linewidth}
    {\bf Englisch:} Fließend \hfill
  \end{minipage}
\end{rSection}

\begin{rSection}{Ehrenamtliches Engagement}

  \hfil  {\bf Maintenance von AUR Paketen}
  \hfill
  {\em 2020 - heute}

  \begin{list}{$\cdot$}{}
  \itemsep -0.5em \vspace{-0.5em}
    \smallskip
  \item Maintenance von einigen \href{https://aur.archlinux.org/packages/?SeB=m&K=jneidel}{Software Paketen} für meine Linux Distribution.
  \end{list}

  {\bf Mitarbeiten an Open Source Projekten}
  \hfill
  {\em 2019 - heute}

  \begin{list}{$\cdot$}{}
  \itemsep -0.5em \vspace{-0.5em}
  \smallskip
  \item Maintenance der größten \href{https://github.com/agarrharr/awesome-cli-apps}{Liste von CLI Apps} auf GitHub.
  \item Erstellung von Bug Reports und Patches für Open Source Software die ich benutze.
  \item Meine eigenen Projekte stelle ich zusammen mit ausreichender Dokumentation der Öffentlichkeit zur Verfügung.
  \end{list}

  \textbf{Organizator des Berliner \href{https://www.freecodecamp.org/}{FreeCodeCamp} Meetups}
  \hfill
  {\em 2018 - 2019}

  \begin{list}{$\cdot$}{}
  \itemsep -0.5em \vspace{-0.5em}
    \smallskip
  \item Leitung eines wöchentlichen Meetups für werdende Entwickler, bei dem ich der Ansprechpartner für technische Fragen war.
  \end{list}
\end{rSection}

\begin{rSection}{Interessen}
    \begin{list}{$\cdot$}{}
    \item Ernährung
    \item Natur
    \item Nachhaltigkeit, Minimalismus
    \item Sport
    \item Schreiben (siehe auch \href{https://jneidel.de}{meine Website})
    \item Programmieren (siehe auch \href{http://github.com/jneidel}{mein GitHub})
    \item Home Sever Administration
    \item Emacs, Automatisierung und Personalisierung meines Computers (siehe auch \href{http://github.com/jneidel/dotfiles}{meine dotfiles})
    \item Note-taking und PKM (Personal Knowledge Management)
  \end{list}
\end{rSection}

\begin{figure}
  \centering
  \includegraphics[width=0.28\linewidth]{img/profile-picture.jpg}
\end{figure}

\end{document}
