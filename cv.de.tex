\documentclass{cv}

\newcommand\verticalPad{1.1cm}
\newcommand\horizontalpad{1.3cm}
\usepackage[left=\horizontalpad,top=\verticalPad,right=\horizontalpad,bottom=\verticalPad]{geometry}
\setlength{\pdfpagewidth}{\paperwidth}
\setlength{\pdfpageheight}{\paperheight}

\name{Jonathan Neidel}
\address{Geboren: {\em7. Aug 1997} in Berlin}
\address{Aktueller Stand: \em{Mar 2024}}
\address{\email{cv@jneidel.com} \\ \github[jneidel] \\ \homepage[jneidel.com]{https://jneidel.com} \\ \linkedin[jneidel]}

\begin{document}

\begin{rSection}{Beschreibung}

  \newcommand\skillWidth{0.48}
  \begin{minipage}[t]{\skillWidth\linewidth}

    Ich bin ein Backend/Fullstack Software Engineer und gut im Lösen von
    Problemen.
    Ich habe aus Neugier mit dem Programmieren begonnen.
    Was mich wirklich überzeugt hat, mich diesem Feld zu widmen, war das
    Potenzial eine Vielzahl von Problemen für mich und andere zu lösen.
    Für mich geht es beim Programmieren vor allem um das Lösen von technischen Problemen.

    \medskip

    Meine Programmier-Expertise kann in zwei sich ergänzende Teile aufgespalten werden.
    Da ist das autodidaktische experimentierende Hacking, angetrieben von dem
    Drang kreative Lösungen zu finden.
    Bestärkt wird dies durch die ausgebildete professionelle Handwerkskunst,
    welche von dem Wunsch getrieben wird Clean Code in einer strukturierten
    kooperativen Manier zu schaffen.

    \medskip

    Vom Anfang an habe ich immer an meinen eigenen Projekten gearbeitet.
    Ich habe alles mögliche gebaut: APIs, ganze Webapps, kleine Scripte und vollumfängliche Kommandozeilen Apps.
    Die Kommandozeilen Umgebung liegt mir besonders am Herzen,
    denn die meiste meiner Arbeit findet im Terminal statt.
    Durch die lange Nutzung meiner selbst konfigurierten Arch Linux
    Umgebung habe ich ein tiefes Wissen über Linux Systeme aufgebaut.

  \end{minipage}
  \hfill
  \begin{minipage}[t]{\skillWidth\linewidth}

    Auf der professionelle Seite habe ich bereits Erfahrung im integrieren
    Software Engineering Workflow gewonnen: Die Arbeit in einer Ticket-basierten
    Umgebung, das Koordinieren mit einem Team von Entwicklern, PMs, QA und anderen
    internen und externen Stakeholdern.
    Ich erstellte qualitative hochwertige Implementationen und kritische Bug-fixes in
    einer fristgerechten Weise.
    Bei all dem habe ich mich immer auf das Verbessern der Umgebung und
    Prozesse, der technischen Dokumentation und der Entwicklungs-Experience
    konzentriert.

    \medskip

    \textbf{Sprachen:}
    \begin{itemize}
      \item Web: JavaScript, TypeScript, SASS, React
      \item Backend/General purpose: Node.js (Express), PHP (Symfony), Python
    \end{itemize}

    \textbf{Datenbanken:}
    \begin{itemize}
      \item SQL Familie
      \item MongoDB
    \end{itemize}

    \textbf{Tools \& Anderes:}
    \begin{itemize}
      \item Linux, Shell scripting (Bash)
      \item Jest, Webpack, Eslint
      \item Git, Gitlab \& GitHub CI
      \item Docker, RabbitMQ
      \item Jira
    \end{itemize}

  \end{minipage}
\end{rSection}

\begin{rSection}{Bildung}

{\bf HTW Berlin}, Berlin \hfill {\em 2019 - 2023}\\
B.Sc in Angewandter Informatik
  \hfill Note: 2,0\\
  \href{https://github.com/jneidel/ba}{Bachelorarbeit: Entwicklung und Evaluation von Methoden zum Verbessern der Nutzbarkeit von CLI Apps}

\smallskip

{\bf Emil Molt Akademie}, Berlin \hfill {\em 2016 - 2018}\\
Fachabitur in Wirtschaft
\hfill NC: 1,3

\end{rSection}

\begin{rSection}{Professionelle Erfahrung}
  \textbf{\href{https://www.jobdirecto.com/}{Jobdirecto} - Fullstack Developer}
  \hfill {\em Feb 2023 - heute}

  Ich habe das Projekt als Lead Developer übernommen.
  Das System und die Infrastruktur wurden aktualisiert; fehlgeleitete Komponente
  refactored; technische Dokumentation geschrieben und ein neuer Backlog von Features abgestimmt.

  \textbf{\href{https://www.endava.de/}{Endava} - Backend Developer}
  \hfill {\em Feb 2022 - heute}

  Als Praktikant fing ich bei Endava an und schrieb anschließend meine Bachelorarbeit im Betrieb.
  Nach dem Uniabschluss nahm ich das Angebot einer vollen Entwickler Stelle an.
  Die zugehörigen Projekte werden unten aufgeführt.

  \textbf{\href{https://buchhaltungsbutler.de/}{Buchhaltungsbutler} - Intern Business Development}
  \hfill {\em Feb - Aug 2017}

  Ein Wirtschaftspraktikum in einem Start-up, bei dem ich die Möglichkeit hatte Businessaufgaben zu automatisieren.
  Außerdem habe ich komplexe Excel Makros zum Umwandlung von Kontoauszügen in das benötigte Format geschrieben.
\end{rSection}

\begin{rSection}{IT Projekte}

  % \textbf{\href{}{}}
  % \hfill

  % \begin{list}{$\cdot$}{}
  %   \itemsep -0.5em \vspace{-0.5em}
  %   \smallskip
  %   \item
  % \end{list}

  \textbf{Endava: \href{https://tvthek.orf.at/}{Big PHP Project}}
  \hfill
  {\em Feb 2023 - heute}

  PHP (Symfony, Sonata), Docker, RabbitMQ, PHPUnit, PHPStan, Webhooks, MariaDB (Doctrine)
  \begin{list}{$\cdot$}{}
  \itemsep -0.5em \vspace{-0.5em}
    \smallskip
  \item Refactoring des Projektes für das Upgrade einer Major PHP Version.
  \item Überarbeitung des lokalen Entwicklungssetups.
  \item Spezifierung und Leitung der Entwicklung eines moderat-großen Features.
  \end{list}

  \textbf{\href{https://www.jobdirecto.com/}{Jobdirecto}}
  \hfill
  {\em Feb 2023 - heute}

  React, Node.js (Express), MongoDB, Webpack, Stripe, Heroku, Jest
  \begin{list}{$\cdot$}{}
  \itemsep -0.5em \vspace{-0.5em}
    \smallskip
  \item Einführung von Testing, technischer Dokumentation, Backups und einer Staging Umgebung.
  \end{list}

  \textbf{Endava: \href{https://github.com/jneidel/oraclett}{oraclett}}
  \hfill
  {\em Dec 2022 - heute}

  Node.js (oclif), Github CI
  gegangen  \begin{list}{$\cdot$}{}
  \itemsep -0.5em \vspace{-0.5em}
    \smallskip
  \item Programmierung einer nutzerfreundlichen CLI App um die Prinzipien meiner Bachelorarbeit zu demonstrieren.
  \end{list}

  \textbf{Endava: Soon@Endava}
  \hfill
  {\em Mar - Nov 2022}

  PHP (Symfony, Sonata), RabbitMQ, Docker
  \begin{list}{$\cdot$}{}
  \itemsep -0.5em \vspace{-0.5em}
    \smallskip
  \item Implementierung einer vollwertigen Webapp basierend auf Symfonys Best-practices.
  \item Bauen eines Microservices zum Versand von Emails basierend auf Message Queue Inputs.
  \item Schreiben der User Stories und Tickets für das Team. Zuteilung von Arbeit und Prüfen von Pull Requests.
  \end{list}

  \textbf{Endava: Changelog Viewer}
  \hfill
  {\em Feb - Mar 2022}

  React, Docker, Gitlab CI
  \begin{list}{$\cdot$}{}
  \itemsep -0.5em \vspace{-0.5em}
    \smallskip
  \item Programmierung eines Parsers und Generators zum Erzeugen einer interaktiven Version der CHANGELOG.md Datei.
  \item Aufsetzen von automatisierten Docker Deployments über Gitlabs CI pipeline.
  \end{list}

  \textbf{\href{https://github.com/htw-kbe-jneidel/main}{Komponenten-basierte Entwicklung - Abschlussprojekt}}
  \hfill
  {\em Dec 2021 - Feb 2022}

  TypeScript, Node.js (Express), RabbitMQ, MongoDB (Mongoose), Github
  \begin{list}{$\cdot$}{}
  \itemsep -0.5em \vspace{-0.5em}
    \smallskip
  \item Implementierung einer Microservice Architektur inklusive RPC Kommunikation via Message Queue.
  \item Schreiben von ausführlicher technischer Dokumentation.
  \end{list}

  \textbf{\href{https://github.com/jneidel/what-to-do}{What to do}}
  \hfill
  {\em Nov 2020 - Apr 2021}

  Node.js (Express), Vue.js, MariaDB, Jest, supertest
  \begin{list}{$\cdot$}{}
  \itemsep -0.5em \vspace{-0.5em}
    \smallskip
  \item Implementierung von Todo App Funktionalität.
  \item Deployment und die Integration der Services.
  \end{list}

  \textbf{\href{https://github.com/vyvytn/avt}{DJ Tool}}
  \hfill
  {\em Oct 2020 - Jan 2021}

  JavaScript, Web Audio API, Node.js (Express), Vue.js
  \begin{list}{$\cdot$}{}
  \itemsep -0.5em \vspace{-0.5em}
    \smallskip
  \item Implementierung dier unterliegende Audiologik.
  \item Verfügbar machen einer externen abgesicherten API über einen Proxy Service.
  \end{list}

  \textbf{\href{https://github.com/jneidel/lock-me-out}{Lock Me Out}}
  \hfill
  {\em Jan - Mar 2019}

  TypeScript, Node.js (Express), GPG, Webpack, SASS, MongoDB, Eslint
  \begin{list}{$\cdot$}{}
  \itemsep -0.5em \vspace{-0.5em}
    \smallskip
  \item Programmierung einer voll ausgestatteten Webapp mit Serverside-rendering (SSR) und einer CLI.
  \item Bauen eines Wrappers um GPG herum um public-key Kryptographie nutzen zu können.
  \end{list}

  \textbf{\href{https://github.com/jneidel/dotfiles}{dotfiles}}
  \hfill
  {\em 2018 - heute}

  Shell scripting, Linux, Shell, vim, tmux
  \begin{list}{$\cdot$}{}
  \itemsep -0.5em \vspace{-0.5em}
    \smallskip
  \item Schreiben von 300+ shell utility scripts für ein breites Feld an Automatisierungsarbeiten.
  \end{list}

  \textbf{\href{https://github.com/jneidel/mangareader-dl}{mangareader-dl}}
  \hfill
  {\em 2018 - heute}

  Node.js, CLI, Web Scraping, TypeScript, TDD, Jest,
  \begin{list}{$\cdot$}{}
  \itemsep -0.5em \vspace{-0.5em}
    \smallskip
  \item Programmierung eines vollumfänglichen Tools zum Herunterladen von Manga.
  \item Neubau der App in TypeScript mit einer klaren OOP Struktur unter einem Test-Driven Ansatz.
  \end{list}

  \textbf{\href{https://github.com/jneidel/setup-webpack}{Setup Webpack}}
  \hfill
  {\em 2018 - heute}

  Webpack, JavaScript, SASS, npm package
  \begin{list}{$\cdot$}{}
  \itemsep -0.5em \vspace{-0.5em}
    \smallskip
  \item Bauen eines Paket was die Funktionalität von Webpack für häufige Use-Cases abstrahiert.
  \item Schreiben von ausführlicher technischer Dokumentation.
  \end{list}

\end{rSection}

\begin{rSection}{Sprachen}
  \begin{minipage}[l]{0.48\linewidth}
    {\bf Deutsch:} Muttersprachler
  \end{minipage}
  \begin{minipage}[l]{0.48\linewidth}
    {\bf Englisch:} Flüssig \hfill
  \end{minipage}
\end{rSection}

\begin{rSection}{Ehrenamtliches Engagement}

  \hfil  {\bf Maintenance von AUR Paketen}
  \hfill
  {\em 2020 - heute}

  \begin{list}{$\cdot$}{}
  \itemsep -0.5em \vspace{-0.5em}
    \smallskip
  \item Maintenance von einigen \href{https://aur.archlinux.org/packages/?SeB=m&K=jneidel}{Software Paketen} für meine Linux Distribution.
  \end{list}

  {\bf Contributing to Open Source projects}
  \hfill
  {\em 2019 - heute}

  \begin{list}{$\cdot$}{}
  \itemsep -0.5em \vspace{-0.5em}
    \smallskip
  \item Maintenance der populärsten \href{https://github.com/agarrharr/awesome-cli-apps}{Liste von CLI Apps} auf GitHub.
  \item Erstellung von Bug Reports und Patches für open-source Software die ich benutze.
  \end{list}

  {\bf Organizator des Berliner FreeCodeCamp Meetups}
  \hfill
  {\em 2018 - 2019}

  \begin{list}{$\cdot$}{}
  \itemsep -0.5em \vspace{-0.5em}
    \smallskip
  \item Veranstaltung eines wöchentlichen Meetups für werdende Entwickler, wo ich der Ansprechpartner für technische Fragen war.
  \end{list}

\end{rSection}

\begin{rSection}{Interessen}

  Neben den Programmieren sind dies einige der Aktivitäten mit denen ich meine Zeit verbringe:

  {\bf Schreiben:}
  Während der Arbeit an meiner Bachelorarbeit merkte ich das mir das Schreiben
  in einer strukturierten Weise irgendwie Spaß macht.
  Diese Energie habe ich in \href{https://jneidel.com/blog/}{meinen Blog}
  gesteckt, wo ich unregelmäßig ausführliche Guides zu Themen schreibe mit denen
  ich mich auseinander gesetzt habe.

  {\bf Zeit in der Natur verbringen:}
  Näher an ``echter" Natur zu sein war einer der Gründe aus Berlin wegzuziehen.
  Ich schätze meine drei-minütige Entfernung zum Wald sehr und nutze diesen täglich.

  {\bf Von Nutzen sein:}
  Ich nehme mir immer die Zeit um ein guter Ehemann/Freund/Bruder/Sohn zu sein.
  Egal welche Form das annimmt.
  Letztens habe ich beispielsweise mein Linux Skillset genutzt um meiner Familie
  ein paar Services auf meinem Home Server zur Verfügung zu stellen.

  {\bf Persönlichkeitsentwicklung:}
  Es bereitet mir Freude mich weiterzuentwickeln und neue Seiten an mir zu entdecken.
  Meine Experimentierfreude ist mir eine große Hilfe dabei.
  Auch nutze ich gerne das Feedback anderer um mich selbst zu reflektieren.

\end{rSection}
\end{document}
