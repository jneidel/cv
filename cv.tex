\documentclass{cv}

\newcommand\verticalPad{1.1cm}
\newcommand\horizontalpad{1.3cm}
\usepackage[left=\horizontalpad,top=\verticalPad,right=\horizontalpad,bottom=\verticalPad]{geometry}
\setlength{\pdfpagewidth}{\paperwidth}
\setlength{\pdfpageheight}{\paperheight}

\name{Jonathan Neidel}
\address{\email{cv@jneidel.com} \\ \homepage[jneidel.com]{https://jneidel.com} \\ \github[jneidel]}
\address{Geboren: {\em7. Aug 1997} in Berlin}


% \address{\href{https://jneidel.com}{https://jneidel.com} \\ \github[jneidel]{https://github.com/jneidel} \\ \href{https://github.com/jneidel}{https://github.com/jneidel}}

\begin{document}

\begin{rSection}{Bildung}

{\bf HTW Berlin}, Berlin \hfill {\em 2019 - Heute}\\
B.Sc in Angewandter Informatik
\hfill 5. Semester

\smallskip

{\bf Emil Molt Akademie}, Berlin \hfill {\em 2016 - 2018}\\
Fachabitur: Wirtschaft
\hfill NC: 1,3

\end{rSection}

\begin{rSection}{IT Skills}

  \newcommand\skillWidth{0.48}
  \begin{minipage}[t]{\skillWidth\linewidth}

Vor dem Studium habe ich auf Node.js als Programmiersprache der Wahl
zurückgegriffen und habe schon alles von APIs, kompletten Webapps, kleinen
Scripten bis hin zu vollständigen Command-Line Apps gebaut. Auch gestützt durch
den gesamten JavaScript Toolstack: Webpack, Jest, Eslint, Vue.js und TypeScript.

    \medskip

Im Laufe des Studiums ich meinen sprachlichen Horizont erweitert und auch
Kontakt mit u.a. Python, Java, Scala und C++ gemacht.

    \medskip

    \textbf{Sprachen:}
    \begin{itemize}
      \item Web: JavaScript, TypeScript, HTML, CSS, SASS
      \item Backend/General: Node.js, Python, Java, Scala
    \end{itemize}
  \end{minipage}
  \hfill
  \begin{minipage}[t]{\skillWidth\linewidth}

Seit 2018 benutze ich Linux als mein primäres Betriebssystem, wobei ich
alles - im Rahmen des möglichen - im Terminal mache. Dabei habe ich mir mein
gesamtes System auf meine Bedürfnisse zurecht geschnitten und dabei Erfahrung
beim Schreiben von diversen Shell Scripts zum Lösen mannigfaltiger Probleme
gesammelt.

Git wird dabei zum Verwalten von allem genutzt und dessen Gebrauch
ist mir schon ins Blut übergegangen.

    \medskip

    \textbf{Tools:}
    \begin{itemize}
      \item Linux, Shell/Bash, Vim
      \item Git, Github
      \item Jira
    \end{itemize}

    \textbf{Datenbanken:}
    \begin{itemize}
      \item PostgreSQL, MariaDB
      \item MongoDB
    \end{itemize}

    % \textbf{Andere:}
    % \begin{itemize}
    %   \item LaTex
    %   \item Web Scraping
    % - Web Scraping
    % - Web Audio API
    % - Chrome Extension
    % - UML
    % \end{itemize}

  \end{minipage}
\end{rSection}

\begin{rSection}{IT Projekte}

  \textbf{\href{https://github.com/jneidel/what-to-do}{what-to-do}}
  \hfill
  {\em Nov 2020 - Apr 2021}

  Node, Vue.js, Express, MariaDB, Jest, supertest

  \begin{list}{$\cdot$}{}
    \itemsep -0.5em \vspace{-0.5em} % Compress items in list together for aesthetics
    \smallskip
    \item Implementation der Todo App Funktionalität
    \item Deployment und Integration
  \end{list}

  \textbf{\href{https://github.com/vyvytn/avt}{DJ Tool}}
  \hfill
  {\em Okt 2020 - Jan 2021}

  JavaScript, Web Audio API, Node, Express
  \begin{list}{$\cdot$}{}
    \itemsep -0.5em \vspace{-0.5em}
    \smallskip
    \item Implementierung der unterliegenden Audio-Logik des Tools
    \item Schreiben der User Stories
    \item Bereitstellung eines Hilfsservers als Schnittstelle zu externen APIs
  \end{list}

  \textbf{\href{https://github.com/jneidel/htw-prog3}{Programmierung 3 Beleg}}
  \hfill
  {\em Okt 2020 - Apr 2021}

  Java, CLI, TCP, UDP, JavaFX
  \begin{list}{$\cdot$}{}
    \itemsep -0.5em \vspace{-0.5em}
    \smallskip
    \item CLI und GUI Interfaces
    \item Kommunikation über TCP/UDP
  \end{list}

  % \textbf{\href{https://github.com/jneidel/yoga-vidya-thunderbird2csv-add-on}{yoga-vidya-thunderbird2csv-add-on}}
  % \hfill
  % {\em Feb 2020}

  % Node, Thunderbird
  % \begin{list}{$\cdot$}{}
  %   \itemsep -0.5em \vspace{-0.5em}
  %   \smallskip
  %   \item Schnittstelle zum Übertragen von Daten aus Emails ins Adressverwaltungsprogramm
  %   \item Dokumentation und Integration im Betrieb
  % \end{list}

  % \textbf{\href{https://github.com/Abrax20/slash-dsgvo}{DSGVO Sync}}
  % \hfill
  % {\em Apr 2019}

  % Node, Express, Microservices, PDF
  % \begin{list}{$\cdot$}{}
  %   \itemsep -0.5em \vspace{-0.5em}
  %   \smallskip
  %   \item Hackathon Contribution
  %   \item API zum generieren einer PDF der Einverständiserklärung
  % \end{list}

  \textbf{\href{https://github.com/jneidel/lock-me-out}{lock-me-out}}
  \hfill
  {\em Jan 2019 - Mär 2019}

  TypeScript, Node, Express, GPG, Webpack, SASS, MongoDB, CLI, Eslint
  \begin{list}{$\cdot$}{}
    \itemsep -0.5em \vspace{-0.5em}
    \smallskip
    \item Webapp mit Serverside-rendering
    \item GPG Wrapper zum verschlüsseln und generieren von Keys
    \item Zugehörige CLI, als alternatives Interface
  \end{list}

  \textbf{\href{https://github.com/jneidel/mangareader-dl}{mangareader-dl}}
  \hfill
  {\em Jan 2018 - Heute}

  Node, CLI, Web Scraping, TDD, Jest, TypeScript
  \begin{list}{$\cdot$}{}
    \itemsep -0.5em \vspace{-0.5em}
    \smallskip
    \item Konfigurierbares Command Line Interface
    \item Scraper zum dynamischen Herunterladen von Bilderreihen
    \item Neubau mit klarer OO-Struktur in TypeScript
    \item Test Driven Development für sicheres Refactoring und zum Überprüfen der
  Funktionalität (bei Server-side Änderungen)
  \end{list}

  \textbf{\href{https://github.com/jneidel/dotfiles}{dotfiles}}
  \hfill
  {\em 2018 - Heute}

  Shell scripting, Linux, Shell, vim, tmux
  \begin{list}{$\cdot$}{}
    \itemsep -0.5em \vspace{-0.5em}
    \smallskip
    \item \textgreater200 Shell Scripts für diverse Anwendungsbereiche
  \end{list}

  \textbf{\href{https://github.com/jneidel/setup-webpack}{setup-webpack}}
  \hfill
  {\em 2018 - Heute}

  Webpack, JavaScript, SASS, npm package
  \begin{list}{$\cdot$}{}
    \itemsep -0.5em \vspace{-0.5em}
    \smallskip
    \item Abstraktion von Webpack für bestimmte Anwendungsfälle
    \item Ausführliche Dokumentation
  \end{list}

  \textbf{\href{https://github.com/jneidel/projects-overview}{projects-overview}}
  \hfill
  {\em Aug 2017 - Mai 2018}

  Node, Express, MongoDB, Webpack, SASS, Mocha
  \begin{list}{$\cdot$}{}
    \itemsep -0.5em \vspace{-0.5em}
    \smallskip
    \item Webapp mit REST API und vollständiger Nutzerverwaltung
  \end{list}

  % \textbf{\href{}{}}
  % \hfill

  % \begin{list}{$\cdot$}{}
  %   \itemsep -0.5em \vspace{-0.5em}
  %   \smallskip
  %   \item
  % \end{list}

\end{rSection}

\begin{rSection}{Ehrenamtliches Engagement}
  {\bf Organisator des Berliner FreeCodeCamp Meetups}
  \hfill
  {\em 2018 - 2019}

  \begin{list}{$\cdot$}{}
    \itemsep -0.5em \vspace{-0.5em}
    \smallskip
    \item Offenes Meetup für Programmieranfänger im Rahmen des FreeCodeCamp Curriculum
    \item Fachlicher und genereller Ansprechpartner für die Teilnehmer
  \end{list}

  {\bf Maintainer der \href{https://github.com/agarrharr/awesome-cli-apps}{awesome-cli-apps} Github Repo}
  \hfill
  {\em 2019 - Heute}

  \begin{list}{$\cdot$}{}
    \itemsep -0.5em \vspace{-0.5em}
    \smallskip
    \item Beliebteste Liste von CLIs auf GitHub, mit 8.7k Stars
    \item Prüfen von Anträgen zum Hinzufügen zur Liste
  \end{list}

  {\bf \href{https://aur.archlinux.org/packages/?SeB=m&K=jneidel}{AUR Package Maintainer}}
  \hfill
  {\em 2020 - Heute}

  \begin{list}{$\cdot$}{}
    \itemsep -0.5em \vspace{-0.5em}
    \smallskip
    \item Zusammenbauen von Pakten für meine Linux Distro
  \end{list}
\end{rSection}

\begin{rSection}{Sprachen}
  \begin{minipage}[l]{0.48\linewidth}
  {\bf Deutsch:} Muttersprachler
  \end{minipage}
  \begin{minipage}[l]{0.48\linewidth}
  {\bf Englisch:} Flüssig \hfill
  \end{minipage}
\end{rSection}

\begin{rSection}{Interessen}
  {\bf Programmieren:} Auch außerhalb des professionellen Kontexts nutze ich meine
Fähigkeiten täglich um alle möglichen Probleme um mich herum zu lösen und mir
das Leben zu erleichtern.

  {\bf Kochen:} Ich esse gerne und genieße es deshalb auch mir etwas frisches,
gesundes zuzubereiten. Außerdem erlaubt es mir mich beim Improvisieren von
Gerichten auszuleben und bietet einen angenehmen Kontrast zum ``vor dem
Bildschirm sitzen''.

  {\bf Sport:} Dieser Kontrast bietet sich auch indem ich mich sportlich auf dem
Fahrrad oder der Yogamatte auslebe. Längere Spaziergänge sorgen auch regelmäßig
für Bewegung.

  {\bf Lesen:} Neben Informatisch-technischen Werken und englischer
Non-Fiction lese ich auch den ein oder anderen Roman.

  {\bf Börse:} Für den Aufbau meiner finanziellen Freiheit setze ich mich mit dem
Investieren auseinander.

  {\bf Meditieren:} Teil der täglichen Routine um das Leben bewusst zu erfahren.
\end{rSection}

\end{document}
