\documentclass{cv}

\newcommand\verticalPad{1.1cm}
\newcommand\horizontalpad{1.3cm}
\usepackage[left=\horizontalpad,top=\verticalPad,right=\horizontalpad,bottom=\verticalPad]{geometry}
\setlength{\pdfpagewidth}{\paperwidth}
\setlength{\pdfpageheight}{\paperheight}

\name{Jonathan Neidel}
\address{\email{cv@jneidel.com} \\ \homepage[jneidel.com]{https://jneidel.com} \\ \github[jneidel]}
\address{Geboren: {\em7. Aug 1997} in Berlin}


% \address{\href{https://jneidel.com}{https://jneidel.com} \\ \github[jneidel]{https://github.com/jneidel} \\ \href{https://github.com/jneidel}{https://github.com/jneidel}}

\begin{document}

\begin{rSection}{Bildung}

{\bf HTW Berlin}, Berlin \hfill {\em 2019 - Heute}\\
B.Sc in Angewandter Informatik

\smallskip

{\bf Emil Molt Akademie}, Berlin \hfill {\em 2016 - 2018}\\
Fachabitur: Wirtschaft \hfill NC: 1,3
% Einschließlich 6 Monaten Praktikum in einem Startup. NC: 1,3
\end{rSection}

\begin{rSection}{IT Skills}
  \newcommand\skillWidth{0.48}
  \begin{minipage}[t]{\skillWidth\linewidth}

Bei der Entwicklung komplexer Lösungen setze ich auf Node.js als Mittel der
Wahl. Ich bin vertraut damit mittels Express APIs und komplette Webapps
umzusetzen, auch gestützt durch den gesamten JavaScript Werkzeugkasten (Webpack
als mächtiges Build-tool, Jest als Test-runner und Eslint für automatisches
Linting) sowie in Verbindung mit Frontend-Frameworks wie Vue.js oder dem
klassenbasierten TypeScript. Dabei liegt mein Fokus auf dem Backend, wo ich Node
auch außerhalb des Web-Kontextes schon zur Verwendung bringen konnte.

    \medskip

    \textbf{Sprachen:}
    \begin{itemize}
      \item JavaScript, TypeScript
      \item HTML, CSS, SASS
      \item Java
    \end{itemize}

    \textbf{Frameworks:}
    \begin{itemize}
      \item Node.js, Express
      \item Vue.js
    \end{itemize}
  \end{minipage}
  \hfill
  \begin{minipage}[t]{\skillWidth\linewidth}

Seit 2018 benutze ich Linux als mein primäres Betriebssystem, wobei ich
alles - im Rahmen des möglichen - im Terminal mache. Dabei habe ich mir mein
gesamtes System (und Tools wie: vim, tmux, etc.) auf meine Bedürfnisse zurecht
geschnitten und dabei Erfahrung beim Schreiben von diversen Shell scripts zum
Lösen mannigfaltiger Probleme gesammelt. Git wird dabei zum verwalten von allem
genutzt und dessen Gebrauch ist mir schon ins Blut übergegangen.

    \medskip

    \textbf{Tools:}
    \begin{itemize}
      \item Linux, Shell/Bash, Vim
      \item Git, Github
      \item Webpack
      \item Jest
      \item Eslint
    \end{itemize}

    \textbf{Datenbanken:}
    \begin{itemize}
      \item MongoDB
      \item PostgreSQL, MariaDB
    \end{itemize}

    % \textbf{Andere:}
    % \begin{itemize}
    %   \item LaTex
    %   \item Web Scraping
    % - Web Scraping
    % - Web Audio API
    % - Chrome Extension
    % - UML
    % \end{itemize}

  \end{minipage}
\end{rSection}

\begin{rSection}{IT Projekte}
  \textbf{\href{https://github.com/jneidel/what-to-do}{what-to-do}}
  \hfill
  {\em Nov 2020 - Apr 2021}

  Node, Vue.js, Express, MariaDB, Jest, supertest

  \begin{list}{$\cdot$}{}
    \itemsep -0.5em \vspace{-0.5em} % Compress items in list together for aesthetics
    \smallskip
    \item Implementation der Todo App Funktionalität
    \item Deployment und Integration
  \end{list}

  \textbf{\href{https://github.com/vyvytn/avt}{DJ Tool}}
  \hfill
  {\em Okt 2020 - Jan 2021}

  JavaScript, Web Audio API, Node, Express
  \begin{list}{$\cdot$}{}
    \itemsep -0.5em \vspace{-0.5em}
    \smallskip
    \item Implementierung der unterliegenden Audio-Logik des Tools
    \item Schreiben der User Stories
    \item Bereitstellung eines Hilfsservers als Schnittstelle zu externen APIs
  \end{list}

  \textbf{\href{https://github.com/jneidel/htw-prog3}{Programmierung 3 Beleg}}
  \hfill
  {\em Okt 2020 - Apr 2021}

  Java, CLI, TCP, UDP, JavaFX
  \begin{list}{$\cdot$}{}
    \itemsep -0.5em \vspace{-0.5em}
    \smallskip
    \item CLI und GUI Interfaces
    \item Kommunikation über TCP/UDP
  \end{list}

  % \textbf{\href{https://github.com/jneidel/yoga-vidya-thunderbird2csv-add-on}{yoga-vidya-thunderbird2csv-add-on}}
  % \hfill
  % {\em Feb 2020}

  % Node, Thunderbird
  % \begin{list}{$\cdot$}{}
  %   \itemsep -0.5em \vspace{-0.5em}
  %   \smallskip
  %   \item Schnittstelle zum Übertragen von Daten aus Emails ins Adressverwaltungsprogramm
  %   \item Dokumentation und Integration im Betrieb
  % \end{list}

  % \textbf{\href{https://github.com/Abrax20/slash-dsgvo}{DSGVO Sync}}
  % \hfill
  % {\em Apr 2019}

  % Node, Express, Microservices, PDF
  % \begin{list}{$\cdot$}{}
  %   \itemsep -0.5em \vspace{-0.5em}
  %   \smallskip
  %   \item Hackathon Contribution
  %   \item API zum generieren einer PDF der Einverständiserklärung
  % \end{list}

  \textbf{\href{https://github.com/jneidel/lock-me-out}{lock-me-out}}
  \hfill
  {\em Jan 2019 - Mär 2019}

  TypeScript, Node, Express, GPG, Webpack, MongoDB, CLI, Eslint
  \begin{list}{$\cdot$}{}
    \itemsep -0.5em \vspace{-0.5em}
    \smallskip
    \item Webapp mit Serverside-rendering
    \item GPG Wrapper zum verschlüsseln und generieren von Keys
    \item Zugehörige CLI, als alternatives Interface
  \end{list}

  \textbf{\href{https://github.com/jneidel/mangareader-dl}{mangareader-dl}}
  \hfill
  {\em Jan 2018 - Heute}

  Node, CLI, Web Scraping, TDD, Jest, TypeScript
  \begin{list}{$\cdot$}{}
    \itemsep -0.5em \vspace{-0.5em}
    \smallskip
    \item Konfigurierbares Command Line Interface
    \item Scraper zum dynamischen Herunterladen von Bilderreihen
    \item Neubau mit klarer OO-Struktur in TypeScript
    \item Test Driven Development für sicheres Refactoring und zum Überprüfen der
  Funktionalität (bei Server-side Änderungen)
  \end{list}

  \textbf{\href{https://github.com/jneidel/dotfiles}{dotfiles}}
  \hfill
  {\em 2018 - Heute}

  Shell scripting, Linux, Shell, vim, tmux
  \begin{list}{$\cdot$}{}
    \itemsep -0.5em \vspace{-0.5em}
    \smallskip
    \item \textgreater200 Shell Scripts für diverse Anwendungsbereiche
  \end{list}

  \textbf{\href{https://github.com/jneidel/setup-webpack}{setup-webpack}}
  \hfill
  {\em 2018 - Heute}

  Webpack, JavaScript, npm package
  \begin{list}{$\cdot$}{}
    \itemsep -0.5em \vspace{-0.5em}
    \smallskip
    \item Abstraktion von Webpack für bestimmte Anwendungsfälle
    \item Ausführliche Dokumentation
  \end{list}

  \textbf{\href{https://github.com/jneidel/projects-overview}{projects-overview}}
  \hfill
  {\em Aug 2017 - Mai 2018}

  Node, Express, MongoDB, Webpack, Mocha
  \begin{list}{$\cdot$}{}
    \itemsep -0.5em \vspace{-0.5em}
    \smallskip
    \item Webapp mit REST API und vollständiger Nutzerverwaltung
  \end{list}

  % \textbf{\href{}{}}
  % \hfill

  % \begin{list}{$\cdot$}{}
  %   \itemsep -0.5em \vspace{-0.5em}
  %   \smallskip
  %   \item
  % \end{list}

\end{rSection}

\begin{rSection}{Ehrenamtliches Engagement}
  {\bf Organisator des Berliner FreeCodeCamp Meetups}
  \hfill
  {\em 2018 - 2019}

  \begin{list}{$\cdot$}{}
    \itemsep -0.5em \vspace{-0.5em}
    \smallskip
    \item Offenes Meetup für Programmieranfänger im Rahmen des FreeCodeCamp Curriculum
    \item Fachlicher und genereller Ansprechpartner für die Teilnehmer
  \end{list}

  {\bf Maintainer der \href{https://github.com/agarrharr/awesome-cli-apps}{awesome-cli-apps} Github Repo}
  \hfill
  {\em 2019 - Heute}

  \begin{list}{$\cdot$}{}
    \itemsep -0.5em \vspace{-0.5em}
    \smallskip
    \item Beliebteste Liste von CLIs auf Github, mit 7,6k Stars
    \item Prüfen von Anträgen zum Hinzufügen zur Liste
  \end{list}
\end{rSection}

\begin{rSection}{Sprachen}
  \begin{minipage}[l]{0.48\linewidth}
  {\bf Deutsch:} Muttersprachler
  \end{minipage}
  \begin{minipage}[l]{0.48\linewidth}
  {\bf Englisch:} Flüssig \hfill
  \end{minipage}
\end{rSection}

\begin{rSection}{Interessen}
  {\bf Programmieren:} Auch außerhalb des professionellen Kontexts nutze ich meine
Fähigkeiten täglich um alle möglichen Probleme um mich herum zu lösen und mir
das Leben zu erleichtern.

  {\bf Kochen:} Ich esse gerne und genieße es deshalb auch mir etwas schönes
zuzubereiten. Außerdem erlaubt es mir mich beim Improvisieren von Gerichten
auszuleben und bietet einen angenehmen Kontrast zum ``vor dem Bildschirm sitzen''.

  {\bf Sport:} Dieser Kontrast bietet sich auch indem ich mich sportlich auf dem
Fahrrad, der Yogamatte oder im Fitnessstudio auslebe.

  {\bf Lesen:} Neben Informatisch-technischen Werken lese ich primär englische
Non-Fiction Bücher aber auch den ein oder anderen Roman.
\end{rSection}

\end{document}
